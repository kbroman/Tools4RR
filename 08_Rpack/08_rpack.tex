\documentclass[12pt,t]{beamer}
\usepackage{graphicx}
\setbeameroption{hide notes}
\setbeamertemplate{note page}[plain]
\usepackage{listings}

% get rid of junk
\usetheme{default}
\beamertemplatenavigationsymbolsempty
\hypersetup{pdfpagemode=UseNone} % don't show bookmarks on initial view


% font
\usepackage{fontspec}
\setsansfont
  [ ExternalLocation = ../fonts/ ,
    UprightFont = *-regular , 
    BoldFont = *-bold ,
    ItalicFont = *-italic ,
    BoldItalicFont = *-bolditalic ]{texgyreheros}
\setbeamerfont{note page}{family*=pplx,size=\footnotesize} % Palatino for notes
% "TeX Gyre Heros can be used as a replacement for Helvetica"
% I've placed them in ../fonts/; alternatively you can install them
% permanently on your system as follows:
%     Download http://www.gust.org.pl/projects/e-foundry/tex-gyre/heros/qhv2.004otf.zip
%     In Unix, unzip it into ~/.fonts
%     In Mac, unzip it, double-click the .otf files, and install using "FontBook"

% named colors
\definecolor{offwhite}{RGB}{255,250,240}
\definecolor{gray}{RGB}{155,155,155}

\ifx\notescolors\undefined % slides
  \definecolor{foreground}{RGB}{255,255,255}
  \definecolor{background}{RGB}{24,24,24}
  \definecolor{title}{RGB}{107,174,214}
  \definecolor{subtitle}{RGB}{102,255,204}
  \definecolor{hilit}{RGB}{102,255,204}
  \definecolor{vhilit}{RGB}{255,111,207}
  \definecolor{lolit}{RGB}{155,155,155}
\else % notes
  \definecolor{background}{RGB}{255,255,255}
  \definecolor{foreground}{RGB}{24,24,24}
  \definecolor{title}{RGB}{27,94,134}
  \definecolor{subtitle}{RGB}{22,175,124}
  \definecolor{hilit}{RGB}{122,0,128}
  \definecolor{vhilit}{RGB}{255,0,128}
  \definecolor{lolit}{RGB}{95,95,95}
\fi
\definecolor{nhilit}{RGB}{128,0,128}  % hilit color in notes
\definecolor{nvhilit}{RGB}{255,0,128} % vhilit for notes

\newcommand{\hilit}{\color{hilit}}
\newcommand{\vhilit}{\color{vhilit}}
\newcommand{\nhilit}{\color{nhilit}}
\newcommand{\nvhilit}{\color{nvhilit}}
\newcommand{\lolit}{\color{lolit}}

% use those colors
\setbeamercolor{titlelike}{fg=title}
\setbeamercolor{subtitle}{fg=subtitle}
\setbeamercolor{institute}{fg=lolit}
\setbeamercolor{normal text}{fg=foreground,bg=background}
\setbeamercolor{item}{fg=foreground} % color of bullets
\setbeamercolor{subitem}{fg=lolit}
\setbeamercolor{itemize/enumerate subbody}{fg=lolit}
\setbeamertemplate{itemize subitem}{{\textendash}}
\setbeamerfont{itemize/enumerate subbody}{size=\footnotesize}
\setbeamerfont{itemize/enumerate subitem}{size=\footnotesize}

% page number
\setbeamertemplate{footline}{%
    \raisebox{5pt}{\makebox[\paperwidth]{\hfill\makebox[20pt]{\lolit
          \scriptsize\insertframenumber}}}\hspace*{5pt}}

% add a bit of space at the top of the notes page
\addtobeamertemplate{note page}{\setlength{\parskip}{12pt}}

% default link color
\hypersetup{colorlinks, urlcolor={hilit}}

\ifx\notescolors\undefined % slides
  % set up listing environment
  \lstset{language=bash,
          basicstyle=\ttfamily\scriptsize,
          frame=single,
          commentstyle=,
          backgroundcolor=\color{darkgray},
          showspaces=false,
          showstringspaces=false
          }
\else % notes
  \lstset{language=bash,
          basicstyle=\ttfamily\scriptsize,
          frame=single,
          commentstyle=,
          backgroundcolor=\color{offwhite},
          showspaces=false,
          showstringspaces=false
          }
\fi

% a few macros
\newcommand{\bi}{\begin{itemize}}
\newcommand{\bbi}{\vspace{24pt} \begin{itemize} \itemsep8pt}
\newcommand{\ei}{\end{itemize}}
\newcommand{\ig}{\includegraphics}
\newcommand{\subt}[1]{{\footnotesize \color{subtitle} {#1}}}
\newcommand{\ttsm}{\tt \small}
\newcommand{\ttfn}{\tt \footnotesize}
\newcommand{\figh}[2]{\centerline{\includegraphics[height=#2\textheight]{#1}}}
\newcommand{\figw}[2]{\centerline{\includegraphics[width=#2\textwidth]{#1}}}



%%%%%%%%%%%%%%%%%%%%%%%%%%%%%%%%%%%%%%%%%%%%%%%%%%%%%%%%%%%%%%%%%%%%%%
% end of header
%%%%%%%%%%%%%%%%%%%%%%%%%%%%%%%%%%%%%%%%%%%%%%%%%%%%%%%%%%%%%%%%%%%%%%

\title{Writing R packages}
\subtitle{Tools for Reproducible Research}
\author{\href{http://www.biostat.wisc.edu/~kbroman}{Karl Broman}}
\institute{Biostatistics \& Medical Informatics, UW{\textendash}Madison}
\date{\href{http://www.biostat.wisc.edu/~kbroman}{\tt \scriptsize \color{foreground} biostat.wisc.edu/{\textasciitilde}kbroman}
\\[-4pt]
\href{http://github.com/kbroman}{\tt \scriptsize \color{foreground} github.com/kbroman}
\\[-4pt]
\href{https://twitter.com/kwbroman}{\tt \scriptsize \color{foreground} @kwbroman}
\\[-4pt]
{\scriptsize Course web: \href{http://bit.ly/tools4rr}{\tt bit.ly/tools4rr}}
}

\begin{document}

{
\setbeamertemplate{footline}{} % no page number here
\frame{
  \titlepage

\note{R packages and the Comprehensive R Archive Network (CRAN) are
  important features of R. R packages provide a simple way to distribute R
  code and documentation. And they really are rather simple to create.

  Write an R package to keep track of the misc.\ R functions
  that you write and reuse. Write an R package to distribute the data
  and software that accompany a paper.

  The most painful part of writing an R package is the construction of
  the documentation files, which are in a special {\tt .Rd}
  format. But the Roxygen2 tool makes this rather easy: you write
  comments preceding each R function, in a specially structured way,
  and then use the Roxygen2 tool to create the {\tt .Rd} files for you.
}
} }




\begin{frame}{Why write an R package?}

\vspace{18pt}

\bbi
\item To distribute R code and documentation
\item To keep track of the misc.\ R functions you write and
  {\hilit reuse}
\item To distribute the data and software that accompany a paper.
\ei

\note{R packages can be big and important. 

     But that shouldn't scare you off. Assembling a few R functions
     within a package will make it vastly easier for {\nhilit you} to
     use them regularly. You don't even need to distribute the package.

     And really, the R package system is an incredibly important
     feature of R. Packages on CRAN are basically guaranteed to be
     installable, as they are regularly built, installed, and tested
     on multiple systems.
}
\end{frame}


\begin{frame}[c]{}

\figh{Figs/R-exts.png}{0.9}

\note{The Writing R Extensions manual is the key source for the
  specifications of R packages. It's rough going in parts, but if you want to
  get a package on CRAN, you should read it.
}
\end{frame}


\begin{frame}{A simple example: RSkittleBrewer}

\vspace{6pt}

\figh{Figs/RSkittleBrewer.png}{0.8}

\vspace{6pt}

\hfill {\tt \footnotesize \lolit \href{http://alyssafrazee.com/RSkittleBrewer.html}{alyssafrazee.com/RSkittleBrewer.html}}


\note{I was going to write a short example R package, but Alyssa
  Frazee saved me the effort. Here package is a perfect little example
  to illustrate how to write a package. 

  It's also a great example of a small but really useful package.
  One small function could be widely useful; you just need to package
  it and tell people about it.
}
\end{frame}


\begin{frame}[fragile]{R package contents}

\vspace{24pt}

\begin{lstlisting}
RSkittleBrewer/

    DESCRIPTION
    NAMESPACE

    R/RSkittleBrewer.R
    R/plotSkittles.R
    R/plotSmarties.R

    man/RSkittleBrewer.Rd
    man/plotSkittles.Rd
    man/plotSmarties.Rd
\end{lstlisting}

\note{
  In the simplest form, an R package is a directory containing: a {\tt DESCRIPTION}
  file (describing the package), a {\tt NAMESPACE} file (indicating
  which functions are available to users), an {\tt R/} directory
  containing R code in {\tt .R} files, and a {\tt man/} directory
  containing documentation in {\tt .Rd} files.
}
\end{frame}

\begin{frame}[fragile]{{\tt DESCRIPTION} file}

\vspace{24pt}

\begin{lstlisting}
Package: RSkittleBrewer
Version: 1.01
Author: Alyssa Frazee
Maintainer: Alyssa Frazee <afrazee@jhsph.edu>
License: MIT + file LICENSE
Title: fun with R colors
Description: for those times you want to make plots with...
URL: https://github.com/alyssafrazee/RSkittleBrewer
\end{lstlisting}

\note{
  The {\tt DESCRIPTION} file is pretty self-explanatory. It just
  contains basic information about the package and its author.

  The simplest way to create this sort of file is to copy and edit one
  from some other package.

  The only part that might be unclear is the {\tt License} field. You
  need to choose a license. We'll talk about this on the very last day
  of the course. 

  For now, I'd suggest choosing between the GPL-3
  (the GNU Public License v3) and MIT licenses. GPL-3 has a
  ``pass-through'' provision: software that incorporates GPL-3 code
  must also be licensed as GPL-3. This is a good but restrictive
  thing. The MIT license is the most bare-bones license possible: it
  basically just says ``Do what you want, but don't blame me.''

  An R package with the MIT license needs to also include a {\tt
  LICENSE} file or R will complain; copy and edit the one from the
  RSkittleBrewer package.
}
\end{frame}



\begin{frame}[fragile]{{\tt NAMESPACE} file}

\vspace{24pt}

\begin{lstlisting}
export(RSkittleBrewer)
export(plotSkittles)
export(plotSmarties)
\end{lstlisting}



\note{
  The {\tt NAMESPACE} file is a bit technical: it tells R what
  functions that will be accessible to users.

  The point of this is to keep different packages from stepping on
  each others' toes.
}
\end{frame}




\begin{frame}[fragile]{An {\tt .Rd} file}

\vspace{12pt}

\small
\begin{lstlisting}
\name{RSkittleBrewer}
\alias{RSkittleBrewer}
\title{Candy-based color palettes}
\description{Vectors of colors corresponding to different
             candies.}
\usage{RSkittleBrewer(flavor = c("original", "tropical", 
                      "wildberry", "M&M", "smarties"))
}
\arguments{
  \item{flavor}{Character string for candy-based color
  palette.}
}
\value{Vector of character strings representing the chosen
       set of colors.}
\examples{
plotSkittles()
plotSmarties()
}
\keyword{hplot}
\seealso{ \code{\link{plotSkittles}},
          \code{\link{plotSmarties}} }
\end{lstlisting}

\note{
  The R documentation format is very LaTeX-like.

  It describes what the function does, what its
  arguments are, and what output it produces.

  You can further provide examples (which can also serve as tests) and
  links to related functions.

  Writing these help files is tedious! That's where Roxygen2 comes in.
}
\end{frame}


\end{document}
