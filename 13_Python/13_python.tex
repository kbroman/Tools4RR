\documentclass[12pt,t]{beamer}
\usepackage{graphicx}
\setbeameroption{hide notes}
\setbeamertemplate{note page}[plain]
\usepackage{listings}

% get rid of junk
\usetheme{default}
\beamertemplatenavigationsymbolsempty
\hypersetup{pdfpagemode=UseNone} % don't show bookmarks on initial view


% font
\usepackage{fontspec}
\setsansfont
  [ ExternalLocation = ../fonts/ ,
    UprightFont = *-regular , 
    BoldFont = *-bold ,
    ItalicFont = *-italic ,
    BoldItalicFont = *-bolditalic ]{texgyreheros}
\setbeamerfont{note page}{family*=pplx,size=\footnotesize} % Palatino for notes
% "TeX Gyre Heros can be used as a replacement for Helvetica"
% I've placed them in ../fonts/; alternatively you can install them
% permanently on your system as follows:
%     Download http://www.gust.org.pl/projects/e-foundry/tex-gyre/heros/qhv2.004otf.zip
%     In Unix, unzip it into ~/.fonts
%     In Mac, unzip it, double-click the .otf files, and install using "FontBook"

% named colors
\definecolor{offwhite}{RGB}{255,250,240}
\definecolor{gray}{RGB}{155,155,155}

\ifx\notescolors\undefined % slides
  \definecolor{foreground}{RGB}{255,255,255}
  \definecolor{background}{RGB}{24,24,24}
  \definecolor{title}{RGB}{107,174,214}
  \definecolor{subtitle}{RGB}{102,255,204}
  \definecolor{hilit}{RGB}{102,255,204}
  \definecolor{vhilit}{RGB}{255,111,207}
  \definecolor{lolit}{RGB}{155,155,155}
\else % notes
  \definecolor{background}{RGB}{255,255,255}
  \definecolor{foreground}{RGB}{24,24,24}
  \definecolor{title}{RGB}{27,94,134}
  \definecolor{subtitle}{RGB}{22,175,124}
  \definecolor{hilit}{RGB}{122,0,128}
  \definecolor{vhilit}{RGB}{255,0,128}
  \definecolor{lolit}{RGB}{95,95,95}
\fi
\definecolor{nhilit}{RGB}{128,0,128}  % hilit color in notes
\definecolor{nvhilit}{RGB}{255,0,128} % vhilit for notes

\newcommand{\hilit}{\color{hilit}}
\newcommand{\vhilit}{\color{vhilit}}
\newcommand{\nhilit}{\color{nhilit}}
\newcommand{\nvhilit}{\color{nvhilit}}
\newcommand{\lolit}{\color{lolit}}

% use those colors
\setbeamercolor{titlelike}{fg=title}
\setbeamercolor{subtitle}{fg=subtitle}
\setbeamercolor{institute}{fg=lolit}
\setbeamercolor{normal text}{fg=foreground,bg=background}
\setbeamercolor{item}{fg=foreground} % color of bullets
\setbeamercolor{subitem}{fg=lolit}
\setbeamercolor{itemize/enumerate subbody}{fg=lolit}
\setbeamertemplate{itemize subitem}{{\textendash}}
\setbeamerfont{itemize/enumerate subbody}{size=\footnotesize}
\setbeamerfont{itemize/enumerate subitem}{size=\footnotesize}

% page number
\setbeamertemplate{footline}{%
    \raisebox{5pt}{\makebox[\paperwidth]{\hfill\makebox[20pt]{\lolit
          \scriptsize\insertframenumber}}}\hspace*{5pt}}

% add a bit of space at the top of the notes page
\addtobeamertemplate{note page}{\setlength{\parskip}{12pt}}

% default link color
\hypersetup{colorlinks, urlcolor={hilit}}

\ifx\notescolors\undefined % slides
  % set up listing environment
  \lstset{language=bash,
          basicstyle=\ttfamily\scriptsize,
          frame=single,
          commentstyle=,
          backgroundcolor=\color{darkgray},
          showspaces=false,
          showstringspaces=false
          }
\else % notes
  \lstset{language=bash,
          basicstyle=\ttfamily\scriptsize,
          frame=single,
          commentstyle=,
          backgroundcolor=\color{offwhite},
          showspaces=false,
          showstringspaces=false
          }
\fi

% a few macros
\newcommand{\bi}{\begin{itemize}}
\newcommand{\bbi}{\vspace{24pt} \begin{itemize} \itemsep8pt}
\newcommand{\ei}{\end{itemize}}
\newcommand{\ig}{\includegraphics}
\newcommand{\subt}[1]{{\footnotesize \color{subtitle} {#1}}}
\newcommand{\ttsm}{\tt \small}
\newcommand{\ttfn}{\tt \footnotesize}
\newcommand{\figh}[2]{\centerline{\includegraphics[height=#2\textheight]{#1}}}
\newcommand{\figw}[2]{\centerline{\includegraphics[width=#2\textwidth]{#1}}}



%%%%%%%%%%%%%%%%%%%%%%%%%%%%%%%%%%%%%%%%%%%%%%%%%%%%%%%%%%%%%%%%%%%%%%
% end of header
%%%%%%%%%%%%%%%%%%%%%%%%%%%%%%%%%%%%%%%%%%%%%%%%%%%%%%%%%%%%%%%%%%%%%%

\title{Python}
\subtitle{Tools for Reproducible Research}
\author{\href{http://www.biostat.wisc.edu/~kbroman}{Karl Broman}}
\institute{Biostatistics \& Medical Informatics, UW{\textendash}Madison}
\date{\href{http://www.biostat.wisc.edu/~kbroman}{\tt \scriptsize \color{foreground} biostat.wisc.edu/{\textasciitilde}kbroman}
\\[-4pt]
\href{http://github.com/kbroman}{\tt \scriptsize \color{foreground} github.com/kbroman}
\\[-4pt]
\href{https://twitter.com/kwbroman}{\tt \scriptsize \color{foreground} @kwbroman}
\\[-4pt]
{\scriptsize Course web: \href{http://bit.ly/tools4rr}{\tt bit.ly/tools4rr}}
}

\begin{document}

{
\setbeamertemplate{footline}{} % no page number here
\frame{
  \titlepage

\note{
  I'm a big proponent of the use of multiple programming languages:
  use different languages for different types of tasks.

  Statisticians, in particular, should be proficient in some
  ``scripting language'' (e.g., Perl, Python, or Ruby). These types of
  languages give you far more flexibility for manipulating data files.

  I've long used Perl, but I've switched to Ruby, and I'm trying to
  also be proficient in Python. I prefer Ruby to Python, but Python is
  much more widely used, and so if you're going to just learn one such
  language, learn Python. (And {\nvhilit don't learn Perl!})
}
} }


\begin{frame}{Why python?}

\bbi
\item Manipulating data files
\item Simulations using others' programs
\onslide<2->{\item Web-related stuff}
\onslide<3->{
\item Alternative to R for data analysis and graphics
\item iPython notebooks
}
\ei

\note{
  For statisticians, the most important use of Python is for the
  manipulation of data files. These scripting languages are great for
  manipulating text, and data files are mostly plain text files.

  In addition, I find a scripting language critical for performing
  simulations to evaluate others' command-line-based programs. They're
  also good for web-related stuff.

  Python can also serve as an alternative to R for data analysis and
  graphics. And iPython notebooks are a big deal for reproducible
  research (and they can be used more broadly than Python).
}
\end{frame}


\begin{frame}{Python 2 vs Python 3}

\bbi
\item Most people are using Python version 2.7
\item Python 3 was introduced in {\vhilit 2006}
  \bi
  \item A number of large changes
  \item Some important Python programs haven't been ported
  \item Few people seem to be using it day-to-day
  \ei
\item You should probably stick with Python 2
\ei

\note{
  The biggest annoyance about Python is the two competing versions,
  Python 2 and Python 3. For now, you should probably stick with
  Python 2.
}
\end{frame}


\begin{frame}{Installing Python}

\bbi
\item On Mac or Unix, Python should be pre-installed
  \bi
  \item[] {\tt python --version}
  \ei
\item For Windows, or to be current, install \href{https://store.continuum.io/cshop/anaconda}{Anaconda}
  \bi
  \item[] Includes NumPy, SciPy, Pandas, iPython, Matplotlib, \dots
  \item[] \href{http://continuum.io/downloads}{\tt
    continuum.io/downloads}
  \ei
\ei

\note{
  When you're just starting to learn, you can just stick with the
  pre-installed version of Python, if you are on some flavor of Unix.

  Long term, I recommend Anaconda, which is an easy-to-install Python
  with basically all of the scientific packages you'd want. Installing
  these by hand seems really painful; installing Anaconda is easy.
}
\end{frame}


\begin{frame}{Functions, modules not scripts}

\note{
}
\end{frame}


\begin{frame}{Unit tests}

\note{
}
\end{frame}

\end{document}
