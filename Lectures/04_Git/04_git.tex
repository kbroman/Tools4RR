\documentclass[12pt,t]{beamer}

\usepackage{graphicx}
\setbeameroption{hide notes}
\setbeamertemplate{note page}[plain]

% get rid of junk
\usetheme{default}
\beamertemplatenavigationsymbolsempty
\hypersetup{pdfpagemode=UseNone} % don't show bookmarks on initial view

% font
\usepackage{fontspec}
\setsansfont{TeX Gyre Heros}
\setbeamerfont{note page}{family*=pplx,size=\footnotesize} % Palatino for notes
% "TeX Gyre Heros can be used as a replacement for Helvetica"
% In Unix, unzip the following into ~/.fonts
% In Mac, unzip it, double-click the .otf files, and install using "FontBook"
%   http://www.gust.org.pl/projects/e-foundry/tex-gyre/heros/qhv2.004otf.zip

% named colors
\definecolor{offwhite}{RGB}{249,242,215}
\definecolor{foreground}{RGB}{255,255,255}
\definecolor{background}{RGB}{24,24,24}
\definecolor{title}{RGB}{107,174,214}
\definecolor{gray}{RGB}{155,155,155}
%\definecolor{darkgray}{RGB}{30,30,30}
\definecolor{subtitle}{RGB}{102,255,204}
\definecolor{hilight}{RGB}{102,255,204}
\definecolor{vhilight}{RGB}{255,111,207}
\definecolor{lolight}{RGB}{155,155,155}
\definecolor{green}{RGB}{125,250,125}

% use those colors
\setbeamercolor{titlelike}{fg=title}
\setbeamercolor{subtitle}{fg=subtitle}
\setbeamercolor{institute}{fg=gray}
\setbeamercolor{normal text}{fg=foreground,bg=background}
\setbeamercolor{item}{fg=foreground} % color of bullets
\setbeamercolor{subitem}{fg=gray}
\setbeamercolor{itemize/enumerate subbody}{fg=gray}
\setbeamertemplate{itemize subitem}{{\textendash}}
\setbeamerfont{itemize/enumerate subbody}{size=\footnotesize}
\setbeamerfont{itemize/enumerate subitem}{size=\footnotesize}

% page number
\setbeamertemplate{footline}{%
    \raisebox{5pt}{\makebox[\paperwidth]{\hfill\makebox[20pt]{\color{gray}
          \scriptsize\insertframenumber}}}\hspace*{5pt}}

% add a bit of space at the top of the notes page
\addtobeamertemplate{note page}{\setlength{\parskip}{12pt}}

% a few macros
\newcommand{\subt}[1]{{\footnotesize \color{subtitle} {#1}}}

\usepackage{listings}

\lstset{language=bash,
        basicstyle=\tiny,
        frame=single,
        backgroundcolor=\color{darkgray},
        commentstyle=\color{green},
        showspaces=false,
        showstringspaces=false
        }

\newcommand{\ft}[1]{\frametitle{#1}}
\newcommand{\bi}{\begin{itemize}}
\newcommand{\bbi}{\vspace{24pt} \begin{itemize} \itemsep8pt}
\newcommand{\ei}{\end{itemize}}
\newcommand{\figw}[2]{\centerline{\includegraphics[width=#2\textwidth]{#1}}}
\newcommand{\figh}[2]{\centerline{\includegraphics[height=#2\textheight]{#1}}}
\newcommand{\fig}[1]{\centerline{\includegraphics{#1}}}

\title[git \& GitHub]{A brief introduction to git \& GitHub}
\author[Broman \& Younkin]{Karl Broman \& Samuel G. Younkin}
\institute{Biostatistics \& Medical Informatics \\ University of Wisconsin{\textendash}Madison}
\date{\href{https://github.com/syounkin/GitPrimer}{\tt \scriptsize github.com/syounkin/GitPrimer}}

\begin{document}
\frame{\titlepage}

\frame{
\vspace{12pt}

% comic from http://www.phdcomics.com/comics/archive.php?comicid=1531
\centerline{\includegraphics[height=3.2in]{Images/phd101212s.png}}

\color{gray} \tiny
\centerline{\url{http://www.phdcomics.com/comics/archive.php?comicid=1531}}
}

\section{Version control}
\frame{
\ft{\only<1>{Methods for tracking versions}\only<2>{Suppose it stops working\dots}}
\bbi
\item Don't keep track
\onslide<2>{
\bi
\item good luck!
\ei
}
\item Save numbered zip files
\onslide<2>{
\bi
\item Unzip versions and {\tt diff}
\ei
}
\item Formal version control
\onslide<2>{
\bi
\item Easy to study changes back in time
\item Easy to jump back and test
\ei
}
\ei
}

\frame{
\ft{Why use formal version control?}
\bbi
\item History of changes
\item Able to go back
\item No worries about breaking things that work
\item Merging changes from multiple people
\ei
}

\frame[c]{
\ft{Example repository}

\figh{Images/example_repo}{0.80}

\onslide<2>{
\vspace*{-0.65\textheight}
\figh{Images/example_repo_zoom}{0.55}
}
}


\frame[c]{
\ft{Example history}

\figh{Images/example_history}{0.80}
}

\frame[c]{
\ft{Example commit}

\figh{Images/example_commit}{0.80}
}




\section{git}

\frame{
\ft{What is git?}
\bbi
\item Formal version control system
\item Developed by Linus Torvalds (developer of Linux)
\bi
\item used to manage the source code for Linux
\ei
\item Tracks any content (but mostly plain text files)
\bi
\item source code
\item data analysis projects
\item manuscripts
\item websites
\item presentations
\ei
\ei
}


\frame{
\ft{Why use git?}
\bbi
\item It's fast
\item You don't need access to a server
\item Amazingly good at merging simultaneous changes
\item Everyone's using it
\ei
}

\section{GitHub}

\frame{
\ft{What is GitHub?}
\bbi
\item A home for git repositories
\item Interface for exploring git repositories
\item {\color{hilight} Real} open source
\bi
\item immediate, easy access to the code
\ei
\item Like facebook for programmers
\item (Bitbucket.org is an alternative)
\bi
\item free private repositories
\ei
\ei
}

\frame{
\ft{Why use GitHub?}
\bbi
\item It takes care of the server aspects of git
\item Graphical user interface for git
\bi
\item Exploring code and its history
\item Tracking issues
\ei
\item Facilitates:
\bi
\item Learning from others
\item Seeing what people are up to
\item Contributing to others' code
\ei
\item Lowers the barrier to collaboration
\bi
\item "There's a typo in your documentation." vs. \\
"Here's a correction for your documentation."
\ei
\ei
}



\section{git Demo}

\begin{frame}
\frametitle{Basic use}

\bbi
\item Change some files
\item See what you've changed
\bi
\item[] {\tt git status}
\item[] {\tt git diff}
\item[] {\tt git log}
\ei
\item Indicate what changes to save
\bi
\item[] {\tt git add}
\ei
\item Commit to those changes
\bi
\item[] {\tt git commit}
\ei
\onslide<2->{
\item Push the changes to GitHub
\bi
\item[] {\tt git push}
\ei }
\onslide<3>{
\item Pull changes from your collaborator
\bi
\item[] {\tt git pull}
\ei }
\ei
\end{frame}

\begin{frame}[fragile]
\frametitle{Initialize repository}
\bbi
\item Create a working directory
\bi
\item For example, \verb|~/GitPrimer|
\ei
\item Initialize it to be a git repository
\bi
\item {\tt \color{hilight} git init}
\item Creates subdirectory \verb|~/GitPrimer/.git|
\ei
\ei
\begin{semiverbatim}
\begin{lstlisting}
$ mkdir ~/GitPrimer
$ cd ~/GitPrimer
$ git init
Initialized empty Git repository in ~/GitPrimer/.git/
\end{lstlisting}
\end{semiverbatim}
\end{frame}

\begin{frame}[fragile]
\frametitle{Produce content}
\bbi
\item Create a README file
\ei
\begin{semiverbatim}
\begin{lstlisting}
Welcome to the GitPrimer repository.

Date: Tue Oct  1 14:12:47 CDT 2013

This repository contains source code for a brief git & GitHub tutorial
given by Younkin & Broman at the University of Wisconsin-Madison,
Dept. of Biostatistics & Medical Informatics.

Email Samuel Younkin <syounkin@stat.wisc.edu> with questions or
comments.
\end{lstlisting}
\end{semiverbatim}
\end{frame}

\begin{frame}[fragile]
\frametitle{Produce content}
\bbi
\item Or create a README.md file
\ei
\begin{semiverbatim}
\begin{lstlisting}
Welcome to the GitPrimer repository.

Date: Tue Oct  1 14:12:47 CDT 2013

This repository contains source code for a brief git & GitHub tutorial
given by [Younkin](http://www.stat.wisc.edu/~syounkin/) &
[Broman](http://www.biostat.wisc.edu/~kbroman) at the University of
Wisconsin-Madison, Dept. of Biostatistics & Medical Informatics.

Email Samuel Younkin <syounkin@stat.wisc.edu> with questions or
comments, or submit an
[Issue at the GitHub](https://github.com/syounkin/GitPrimer/issues).
\end{lstlisting}
\end{semiverbatim}
\end{frame}

\begin{frame}[fragile]
\frametitle{Incorporate into repository}
\bbi
\item Stage the changes using {\tt \color{hilight} git add}
\ei
\begin{semiverbatim}
\begin{lstlisting}
$ git add README
\end{lstlisting}
\end{semiverbatim}
\end{frame}

\begin{frame}[fragile]
\frametitle{Incorporate into repository}
\bbi
\item Now commit using {\tt \color{hilight} git commit}
\ei
\begin{semiverbatim}
\begin{lstlisting}
$ git commit -m "Initial commit of README file"
[master (root-commit) 32c9d01] Initial commit of README file
 1 file changed, 14 insertions(+)
 create mode 100644 README
\end{lstlisting}
\end{semiverbatim}

\bi
\item The \texttt{-m} argument allows one to enter a message
\item Without \texttt{-m}, \texttt{git} will spawn a text editor
\item Use a meaningful message
\item Message can have multiple lines, but make 1st line an overview
\ei
\end{frame}




\begin{frame}[fragile]
\ft{A few points on commits}
\bbi
\item Use frequent, small commits
\item Don't get out of sync with your collaborators
\item Commit the sources, not the derived files
\bi
\item[] (R code not images)
\ei
\item Use a {\tt .gitignore} file to indicate files to be ignored
\ei


\begin{semiverbatim}
\begin{lstlisting}
*~
manuscript.pdf
Figs/*.pdf
.RData
.RHistory
*.Rout
*.aux
*.log
*.out
\end{lstlisting}
\end{semiverbatim}


\end{frame}


\begin{frame}[fragile]
\ft{Removing/moving files}

\vspace{24pt}

For files that are being tracked by git:

\bigskip

\hspace{1em} Use {\tt \color{hilight} git rm} instead of just {\tt rm}

\hspace{1em} Use {\tt \color{hilight} git mv} instead of just {\tt mv}

\begin{semiverbatim}
\begin{lstlisting}
$ git rm myfile
$ git mv myfile newname
$ git mv myfile SubDir/
$ git commit
\end{lstlisting}
\end{semiverbatim}

\end{frame}



\frame{
\ft{Using git on an existing project}

\bbi
\item {\tt git init}
\item Set up {\tt .gitignore} file
\item {\tt git status} {\footnotesize \color{lolight} (did you miss any?)}
\item {\tt git add .} {\footnotesize \color{lolight} (or name files individually)}
\item {\tt git status} {\footnotesize \color{lolight} (did you miss any?)}
\item {\tt git commit}
\ei

}


\begin{frame}
\frametitle{Basic use}

\bbi
\item Change some files
\item See what you've changed
\bi
\item[] {\tt git status}
\item[] {\tt git diff}
\item[] {\tt git log}
\ei
\item Indicate what changes to save
\bi
\item[] {\tt git add}
\ei
\item Commit to those changes
\bi
\item[] {\tt git commit}
\ei
\item Push the changes to GitHub
\bi
\item[] {\tt git push}
\ei
\item Pull changes from your collaborator
\bi
\item[] {\tt git pull}
\ei
\ei
\end{frame}




\section{GitHub Demo}

\begin{frame}[fragile]
\ft{Getting started with GitHub}
\bbi
\item Get an account
\item Set up ssh keys
\bi
\item Look for files {\color{hilight} \verb|~/.ssh/id_rsa|} and {\color{hilight} \verb|~/.ssh/id_rsa.pub|}
\item {\color{hilight} \verb|ssh-keygen |}\verb|-t rsa -C "your_email@example.com"|
\item Copy contents of \verb|~/.ssh/id_rsa.pub|
\ei
\item Add SSH key at GitHub
\bi
\item Account settings
\item SSH Keys
\item Add SSH key
\item Paste contents of \verb|~/.ssh/id_rsa.pub|
\ei
\item Similar thing at BitBucket
\ei
\end{frame}

\frame{
\ft{Set up GitHub repository}

\only<1>{
\bbi
\item Click the "Create a new repo" button
\item Give it a {\color{hilight} name} and description
\item Click the "Create repository" button
\item Back at the command line:
  \bi
  \item[] {\tt git remote add origin git@github.com:username/{\color{hilight} repo}}
  \item[] {\tt git push -u origin master}
  \ei
\ei
}
\only<2->{\vspace{24pt}}
\only<2>{\figw{Images/new_repo_1.png}{0.95}}
\only<3>{\figw{Images/new_repo_2.png}{0.95}}

}

\begin{frame}[fragile]
\ft{Configuration file}

\vspace{24pt}

Part of a {\tt .git/config} file:

\begin{semiverbatim}
\begin{lstlisting}
[remote "origin"]
	url = git@github.com:kbroman/qtl.git
	fetch = +refs/heads/*:refs/remotes/origin/*

[branch "master"]
	remote = origin
	merge = refs/heads/master

[remote "brian"]
	url = git://github.com/byandell/qtl.git
	fetch = +refs/heads/*:refs/remotes/brian/*
\end{lstlisting}
\end{semiverbatim}

\end{frame}


\frame{
\ft{Issues and pull requests}
\bbi
\item Problem with or suggestion for someone's code?
\bi
\item Point it out as an Issue
\ei
\item Even better: Provide a fix
\bi
\item Fork
\item Clone
\item Modify
\item Commit
\item Push
\item Submit a Pull Request
\ei
\ei
}


\begin{frame}[fragile]
\ft{Suggest a change to a repo}
\bbi
\item Go to the repository:
\bi
\item[] \verb|http://github.com/someone/repo|
\ei
\item {\color{hilight} Fork} the repository
\bi
\item[] Click the "Fork" button
\ei
\item {\color{hilight} Clone} your version of it
\bi
\item[] {\tt git clone git@github.com:username/repo}
\ei
\item Change things locally, {\tt git \color{hilight} add}, {\tt git \color{hilight} commit}
\item Push your changes to \emph{your\/} GitHub repository
\bi
\item[] {\tt git \color{hilight} push}
\ei
\item Go to \emph{your\/} GitHub repository
\item Click "{\color{hilight} Pull Requests}" and "New pull request"
\ei
\end{frame}


\begin{frame}[fragile]
\ft{Pulling a friend's changes}
\bbi
\item Add a connection
\bi
\item[] {\tt git remote add friend git://github.com/friend/repo}
\ei
\item Pull the changes
\bi
\item[] {\tt git pull friend master}
\ei
\item Push them back to your GitHub repo
\bi
\item[] {\tt git push}
\ei
\ei
\end{frame}


\begin{frame}[fragile]
\ft{Merge conflicts}

\vspace{24pt}

Sometimes after {\color{hilight} \tt git pull friend master}

\begin{semiverbatim}
\begin{lstlisting}
Auto-merging README.md
CONFLICT (content): Merge conflict in README.md
Automatic merge failed; fix conflicts and then commit the result.
\end{lstlisting}
\end{semiverbatim}

Inside the file you'll see:
\begin{semiverbatim}
\begin{lstlisting}
<<<<<<< HEAD
A line in my file.
=======
A line in my friend's file
>>>>>>> 031389f2cd2acde08e32f0beb084b2f7c3257fff
\end{lstlisting}
\end{semiverbatim}

Edit, add, commit, push, submit pull request.
\end{frame}


\begin{frame}[c]
\ft{git/GitHub with RStudio}

\only<1>{\figw{Images/RStudio01.png}{0.90}}
\only<2>{\figh{Images/RStudio02.png}{0.60}}
\only<3>{\figh{Images/RStudio03.png}{0.60}}
\only<4>{\figw{Images/RStudio04.png}{0.90}}
\only<5>{\figw{Images/RStudio05.png}{0.90}}
\only<6>{\figw{Images/RStudio06.png}{0.90}}
\only<7>{\figw{Images/RStudio04.png}{0.90}}
\only<8>{\figh{Images/RStudio07.png}{0.80}}
\only<9>{\figh{Images/RStudio08.png}{0.80}}
\only<10>{\figw{Images/RStudio09.png}{0.90}}
\only<11>{
\tt git remote add origin git@github.com:kbroman/repo

git push -u origin master
}
\only<12>{\figw{Images/RStudio10.png}{0.90}}
\only<13>{\figh{Images/RStudio11.png}{0.80}}

\end{frame}

\frame[c]{
\ft{Delete GitHub repo}

\figh{Images/RStudio12.png}{0.80}
}


\section{Learn More}
\frame{
\ft{}

\vspace{25mm}

Open source means everyone can see my stupid mistakes.

\vspace{5mm}

Version control means everyone can see every stupid mistake I've ever
made.

\vspace{33mm}
\centerline{\scriptsize \tt \color{gray} \href{http://bit.ly/stupidcode}{bit.ly/stupidcode}}
}

\frame{
\ft{Resources}
\bbi
\item Look at others' repositories:
\bi
\item Hadley Wickham (ggplot2):
{\footnotesize \url{https://github.com/hadley}}
\item Yihui Xie (knitr):
{\footnotesize \url{https://github.com/yihui}}
\ei

\item Karl's tutorial:
{\color{gray} \scriptsize \url{http://kbroman.github.io/github_tutorial}}

\item Karthik Ram's slides:
{\color{gray} \scriptsize \url{http://karthikram.github.io/git_intro}}

\item Pro Git book: 
{\color{gray} \scriptsize \url{http://git-scm.com/book}}
\ei
}

\end{document}
