\documentclass[12pt,t]{beamer}
\usepackage{graphicx}
\setbeameroption{hide notes}
\setbeamertemplate{note page}[plain]
\usepackage{listings}

% set up listing environment
\lstset{language=bash,
        basicstyle=\scriptsize,
        frame=single,
        backgroundcolor=\color{darkgray},
        commentstyle=\color{green},
        showspaces=false,
        showstringspaces=false
        }

% get rid of junk
\usetheme{default}
\beamertemplatenavigationsymbolsempty
\hypersetup{pdfpagemode=UseNone} % don't show bookmarks on initial view

% font
\usepackage{fontspec}
\setsansfont{TeX Gyre Heros}
\setbeamerfont{note page}{family*=pplx,size=\footnotesize} % Palatino for notes
% "TeX Gyre Heros can be used as a replacement for Helvetica"
% In Unix, unzip the following into ~/.fonts
% In Mac, unzip it, double-click the .otf files, and install using "FontBook"
%   http://www.gust.org.pl/projects/e-foundry/tex-gyre/heros/qhv2.004otf.zip

% named colors
\definecolor{offwhite}{RGB}{249,242,215}
\definecolor{foreground}{RGB}{255,255,255}
\definecolor{background}{RGB}{24,24,24}
\definecolor{title}{RGB}{107,174,214}
\definecolor{gray}{RGB}{155,155,155}
\definecolor{subtitle}{RGB}{102,255,204}
\definecolor{hilight}{RGB}{102,255,204}
\definecolor{vhilight}{RGB}{255,111,207}
\definecolor{nhilight}{RGB}{128,0,128}  % hilight color in notes
\definecolor{nvhilight}{RGB}{255,0,128} % vhilight for notes
\definecolor{lolight}{RGB}{155,155,155}
%\definecolor{green}{RGB}{125,250,125}

% use those colors
\setbeamercolor{titlelike}{fg=title}
\setbeamercolor{subtitle}{fg=subtitle}
\setbeamercolor{institute}{fg=gray}
\setbeamercolor{normal text}{fg=foreground,bg=background}
\setbeamercolor{item}{fg=foreground} % color of bullets
\setbeamercolor{subitem}{fg=gray}
\setbeamercolor{itemize/enumerate subbody}{fg=gray}
\setbeamertemplate{itemize subitem}{{\textendash}}
\setbeamerfont{itemize/enumerate subbody}{size=\footnotesize}
\setbeamerfont{itemize/enumerate subitem}{size=\footnotesize}

% page number
\setbeamertemplate{footline}{%
    \raisebox{5pt}{\makebox[\paperwidth]{\hfill\makebox[20pt]{\color{lolight}
          \scriptsize\insertframenumber}}}\hspace*{5pt}}

% add a bit of space at the top of the notes page
\addtobeamertemplate{note page}{\setlength{\parskip}{12pt}}

% a few macros
\newcommand{\bi}{\begin{itemize}}
\newcommand{\ei}{\end{itemize}}
\newcommand{\ig}{\includegraphics}
\newcommand{\subt}[1]{{\footnotesize \color{subtitle} {#1}}}
\newcommand{\ttsm}{\tt \small}

%%%%%%%%%%%%%%%%%%%%%%%%%%%%%%%%%%%%%%%%%%%%%%%%%%%%%%%%%%%%%%%%%%%%%%
% end of header
%%%%%%%%%%%%%%%%%%%%%%%%%%%%%%%%%%%%%%%%%%%%%%%%%%%%%%%%%%%%%%%%%%%%%%

% title info
\title{Unix command line; editors}
\subtitle{Tools for Reproducible Research}
\author{\href{http://www.biostat.wisc.edu/~kbroman}{Karl Broman}}
\institute{Biostatistics \& Medical Informatics, UW{\textendash}Madison}
\date{\href{http://www.biostat.wisc.edu/~kbroman}{\tt \scriptsize biostat.wisc.edu/{\textasciitilde}kbroman}
\\[-4pt]
\href{http://github.com/kbroman}{\tt \scriptsize github.com/kbroman}
\\[-4pt]
\href{https://twitter.com/kwbroman}{\tt \scriptsize @kwbroman}
\\[-4pt]
{\scriptsize Course web: \href{http://bit.ly/tools4rr}{\tt \color{hilight} bit.ly/tools4rr}}
}


\begin{document}

% title slide
{
\setbeamertemplate{footline}{} % no page number here
\frame{
  \titlepage
  \note{For your work to be reproducible, it needs to be code-based;
    don't touch that mouse!

Some flavor of Unix will be most efficient; you'll want to be
comfortable with command-line tools.

And you'll spend a lot of time using an editor. Learn to use a
powerful one.
} }


\begin{frame}[c]{}


\centering
\Large

Windows {\color{lolight} vs.} Mac OSX {\color{lolight} vs.} Linux

\bigskip
\bigskip

Remote {\color{lolight} vs.} Not

\note{The Windows operating system is not very programmer-friendly.

Mac OSX isn't either, but under the hood, it's just unix.

Don't touch the mouse! Open a terminal window and start typing.

I do most of my work directly on my desktop or laptop. You might
prefer to work remotely on a server, instead. But I can't stand having
any lag in looking at graphics.
}
\end{frame}


\begin{frame}[c]{If you're stuck with Windows...}

\centering
\Large

Consider {\color{hilight} \href{http://www.cygwin.org}{Cygwin}}
{\color{lolight} (and perhaps {\color{hilight} \href{https://code.google.com/p/mintty/}{Mintty}})}

\note{Cygwin is an effort to get Unix command-line tools in Windows.

Mintty is a terminal emulator.}
\end{frame}

\begin{frame}[c]{If you use a Mac...}

\centering
\Large

Consider {\color{hilight} \href{http://brew.sh/}{Homebrew}} and
{\color{hilight} \href{http://www.iterm2.com}{iTerm2}}

\note{Homebrew is a packaging system; iTerm2 is a Terminal replacement}

\end{frame}




\end{document}
