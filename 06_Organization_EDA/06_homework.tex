\documentclass[12pt]{article}

\usepackage{times}
\usepackage{xcolor}
\usepackage{hyperref}

\hypersetup{pdfpagemode=UseNone} % don't show bookmarks on initial view
\definecolor{hilit}{RGB}{122,0,128}
\hypersetup{colorlinks, urlcolor={hilit}}
\newcommand{\ttsm}{\tt \small}


% revise margins
\setlength{\headheight}{0.0in}
\setlength{\topmargin}{-0.5in}
\setlength{\headsep}{0.0in}
\setlength{\textheight}{10in}
\setlength{\footskip}{0.0in}
\setlength{\oddsidemargin}{0.0in}
\setlength{\evensidemargin}{0.0in}
\setlength{\textwidth}{6.5in}

\setlength{\parskip}{6pt}
\setlength{\parindent}{0pt}

\begin{document}

\thispagestyle{empty}

\textbf{Tools for Reproducible Research} \\
Homework, 6 March 2015

\bigskip

\begin{enumerate}

\item Look at the file organization for an old project (or a current
  project, if you don't have any old ones).

  \begin{enumerate}
  \item Is it easy to understand what everything is?
  \item Did you document the process of data preparation and
    analysis?
  \item How might you reorganize things, to make it more clear to
    someone else, or to you a year from now?
  \item Do you need more documentation (e.g., {\ttsm ReadMe} files) or
    are the file names and organization self-explanatory?
  \end{enumerate}

\item Try out \href{http://yihui.name/knitr/demo/stitch/}{knitr spin}.

  \begin{enumerate}
    \item Look at
      \href{https://github.com/kbroman/Tools4RR/tree/master/06_Organization_EDA/Examples}{the
        example for this week's lecture}.
    \item Adapt one of your own recent R scripts to use {\ttsm
      spin}-style comments.
  \end{enumerate}

  Does this seem useful for capturing exploratory data analysis?
\end{enumerate}

\end{document}
