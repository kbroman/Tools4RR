\documentclass[12pt,t]{beamer}
\usepackage{graphicx}
\setbeameroption{hide notes}
\setbeamertemplate{note page}[plain]
\usepackage{listings}

% set up listing environment
\lstset{language=bash,
        basicstyle=\ttfamily\scriptsize,
        frame=single,
        commentstyle=,
        backgroundcolor=\color{darkgray},
        showspaces=false,
        showstringspaces=false
        }

% get rid of junk
\usetheme{default}
\beamertemplatenavigationsymbolsempty
\hypersetup{pdfpagemode=UseNone} % don't show bookmarks on initial view


% font
\usepackage{fontspec}
\setsansfont{TeX Gyre Heros}
\setbeamerfont{note page}{family*=pplx,size=\footnotesize} % Palatino for notes
% "TeX Gyre Heros can be used as a replacement for Helvetica"
% In Unix, unzip the following into ~/.fonts
% In Mac, unzip it, double-click the .otf files, and install using "FontBook"
%   http://www.gust.org.pl/projects/e-foundry/tex-gyre/heros/qhv2.004otf.zip

% named colors
\definecolor{offwhite}{RGB}{249,242,215}
\definecolor{foreground}{RGB}{255,255,255}
\definecolor{background}{RGB}{24,24,24}
\definecolor{title}{RGB}{107,174,214}
\definecolor{gray}{RGB}{155,155,155}
\definecolor{subtitle}{RGB}{102,255,204}
\definecolor{hilit}{RGB}{102,255,204}
\definecolor{vhilit}{RGB}{255,111,207}
\definecolor{nhilit}{RGB}{128,0,128}  % hilit color in notes
\definecolor{nvhilit}{RGB}{255,0,128} % vhilit for notes
\definecolor{lolit}{RGB}{155,155,155}

\newcommand{\hilit}{\color{hilit}}
\newcommand{\vhilit}{\color{vhilit}}
\newcommand{\nhilit}{\color{nhilit}}
\newcommand{\nvhilit}{\color{nvhilit}}
\newcommand{\lolit}{\color{lolit}}

% use those colors
\setbeamercolor{titlelike}{fg=title}
\setbeamercolor{subtitle}{fg=subtitle}
\setbeamercolor{institute}{fg=gray}
\setbeamercolor{normal text}{fg=foreground,bg=background}
\setbeamercolor{item}{fg=foreground} % color of bullets
\setbeamercolor{subitem}{fg=gray}
\setbeamercolor{itemize/enumerate subbody}{fg=gray}
\setbeamertemplate{itemize subitem}{{\textendash}}
\setbeamerfont{itemize/enumerate subbody}{size=\footnotesize}
\setbeamerfont{itemize/enumerate subitem}{size=\footnotesize}

% page number
\setbeamertemplate{footline}{%
    \raisebox{5pt}{\makebox[\paperwidth]{\hfill\makebox[20pt]{\lolit
          \scriptsize\insertframenumber}}}\hspace*{5pt}}

% add a bit of space at the top of the notes page
\addtobeamertemplate{note page}{\setlength{\parskip}{12pt}}

% default link color
\hypersetup{colorlinks, urlcolor={hilit}}

% a few macros
\newcommand{\bi}{\begin{itemize}}
\newcommand{\bbi}{\vspace{24pt} \begin{itemize} \itemsep8pt}
\newcommand{\ei}{\end{itemize}}
\newcommand{\ig}{\includegraphics}
\newcommand{\subt}[1]{{\footnotesize \color{subtitle} {#1}}}
\newcommand{\ttsm}{\tt \small}
\newcommand{\figh}[2]{\centerline{\includegraphics[height=#2\textheight]{#1}}}
\newcommand{\figw}[2]{\centerline{\includegraphics[width=#2\textwidth]{#1}}}

%%%%%%%%%%%%%%%%%%%%%%%%%%%%%%%%%%%%%%%%%%%%%%%%%%%%%%%%%%%%%%%%%%%%%%
% end of header
%%%%%%%%%%%%%%%%%%%%%%%%%%%%%%%%%%%%%%%%%%%%%%%%%%%%%%%%%%%%%%%%%%%%%%

\title{Tools for Reproducible Research}
\subtitle{Organizing projects; exploratory data analysis}
\author{\href{http://www.biostat.wisc.edu/~kbroman}{Karl Broman}}
\institute{Biostatistics \& Medical Informatics, UW{\textendash}Madison}
\date{\href{http://www.biostat.wisc.edu/~kbroman}{\tt \scriptsize \color{white} biostat.wisc.edu/{\textasciitilde}kbroman}
\\[-4pt]
\href{http://github.com/kbroman}{\tt \scriptsize \color{white} github.com/kbroman}
\\[-4pt]
\href{https://twitter.com/kwbroman}{\tt \scriptsize \color{white} @kwbroman}
\\[-4pt]
{\scriptsize Course web: \href{http://bit.ly/tools4rr}{\tt bit.ly/tools4rr}}
}

\begin{document}

{
\setbeamertemplate{footline}{} % no page number here
\frame{
  \titlepage

\note{I'm trying to cover two things here: how to organize data
  analysis projects, so in the end the results will be reproducible
  and clear, and how to capture the results of exploratory data
  analysis.

  The hardest part, regarding organizing projects, concerns how to coordinate
  collaborative projects: to keep data, code, and results synchronized
  among collaborators.

  Regarding exploratory data analysis, we want to capture the whole
  process: what you're trying to do, what you're thinking about, what
  you're seeing, and what you're concluding and why. And we want to do
  so without getting in the way of the creative process.

  I'll sketch what I try to do, and the difficulties I've had. But I
  don't have all of the answers.
}
} }


\begin{frame}[fragile]{Organizing your stuff}

\vspace{6pt}

\begin{lstlisting}
Code/d3examples/
    /Others/
    /PyBroman/
    /Rbroman/
    /Rqtl/
    /Rqtlcharts/
Docs/Talks/
    /Meetings/
    /Others/
    /Papers/
    /Resume/
    /Reviews/
    /Travel/
Play/
Projects/AlanAttie/
        /BruceTempel/
        /Hassold_QTL/
        /Hassold_Age/
        /Payseur_Gough/
        /PhyloQTL/
        /Tar/
\end{lstlisting}

\note{This is basically how I organize my harddrive. You want it to be
  clear where things are. You shouldn't be searching for stuff.

  In my {\tt Projects/} directory, I have a {\tt Tar/} directory with
  {\tt tar.gz} files
  of older projects; the same is true for other directories, like
  {\tt Docs/Papers/} and {\tt Docs/Talks/}.
}
\end{frame}


\begin{frame}[fragile]{Organizing your projects}

\vspace{6pt}

\begin{lstlisting}
Projects/Hassold_QTL/

    Data/
    Notes/
    R/
    R/Figs/
    R/Cache/
    Rawdata/
    Refs/

    Makefile
    Readme.txt

    Python/convertGeno.py
    Python/convertPheno.py
    Python/combineData.py

    R/prepData.R
    R/analysis.R
    R/diagnostics.Rmd
    R/qtl_analysis.Rmd
\end{lstlisting}

\note{This is how I'd organize a simple project.

  Separate the raw data from processed data.

  Separate code from data.

  Include a Readme file and a Makefile.

  I tend to reuse file names. Almost every project will have an {\tt
    R/prepData.R} script.

  Of course, each project is under version control (with git)!

  {\tt R/analysis.R} usually has exploratory analyses, and then
  there'll be separate {\tt .Rmd} files with more finalized work.
}
\end{frame}


\begin{frame}{Basic principles}

\vspace{6pt}

\bi
\item Develop your own system
\item Put everything in a common directory
\item Be consistent
\bi
\item directory structure; names
\ei
\item Separate raw from processed data
\item Separate code from data
\item It should be obvious what code created what files, and what the
  dependencies were.
\item No hand-editing of data files
\item Don't put spaces in file names!
\item Use relative paths, not absolute paths
\bi
\item {\tt ../blah} not {\tt {\textasciitilde}/blah} or {\tt /users/blah}
\ei
\ei

\note{I work on many different projects at the same time, and I'll
  come back to a project 6 months or a year later.

  You shouldn't have to spend much time figuring out where things are
  and how things were created: have a {\tt Makefile}, and keep notes. But
  notes are not necessarily correct while a {\tt Makefile} would be.

  Plan for the whole deal to ultimately be open to others: will you be
  proud of the work, or embarrassed by the mess?
}
\end{frame}


\begin{frame}[c]{}

\centering
\large
Your closest collaborator is you six months ago, but you
don't reply to emails.
\note{I heard this from Paul Wilson, UW-Madison.
}
\end{frame}


\begin{frame}{Painful bits}

\vspace{6pt}

\bi
\item Coming up with good names for things
\bi
\item Code as verbs; data as nouns
\ei
\item Stages of data cleaning
\item Going back and redoing stuff
\item Clutter of old stuff that you no longer need
\item Keeping track of the order of things
\bi
\item dependencies; what gave rise to what
\ei
\item Long, messy Makefiles
\ei

\note{
}
\end{frame}


\begin{frame}{Problem: 80 million side projects}

\vspace{6pt}


\note{Show what's happened with the Attie project.
}
\end{frame}



\begin{frame}{Saving intermediate results}

\vspace{6pt}


\note{Sub-folders can be their own git repositories.
}
\end{frame}




\begin{frame}{Problem: Coordinating with collaborators}

\vspace{6pt}


\note{Syncing data, derived data, code.

  Divvying up tasks.
}
\end{frame}






\begin{frame}{Problem: Collaborators who don't use git}

\vspace{6pt}


\note{Use it yourself, and copy over your stuff.
}
\end{frame}




\begin{frame}{Problem: Variations in names across files}

\vspace{6pt}


\note{Create a separate file with meta-data: ``In this file, the
  variable is called blah while in that file it's blather.''

  Similarly, complications in the names of files.
}
\end{frame}




\end{document}
