\documentclass[12pt,t]{beamer}
\usepackage{graphicx}
\setbeameroption{hide notes}
\setbeamertemplate{note page}[plain]
\usepackage{listings}

% set up listing environment
\lstset{language=bash,
        basicstyle=\ttfamily\scriptsize,
        frame=single,
        commentstyle=,
        backgroundcolor=\color{darkgray},
        showspaces=false,
        showstringspaces=false
        }

% get rid of junk
\usetheme{default}
\beamertemplatenavigationsymbolsempty
\hypersetup{pdfpagemode=UseNone} % don't show bookmarks on initial view


% font
\usepackage{fontspec}
\setsansfont{TeX Gyre Heros}
\setbeamerfont{note page}{family*=pplx,size=\footnotesize} % Palatino for notes
% "TeX Gyre Heros can be used as a replacement for Helvetica"
% In Unix, unzip the following into ~/.fonts
% In Mac, unzip it, double-click the .otf files, and install using "FontBook"
%   http://www.gust.org.pl/projects/e-foundry/tex-gyre/heros/qhv2.004otf.zip

% named colors
\definecolor{offwhite}{RGB}{249,242,215}
\definecolor{foreground}{RGB}{255,255,255}
\definecolor{background}{RGB}{24,24,24}
\definecolor{title}{RGB}{107,174,214}
\definecolor{gray}{RGB}{155,155,155}
\definecolor{subtitle}{RGB}{102,255,204}
\definecolor{hilit}{RGB}{102,255,204}
\definecolor{vhilit}{RGB}{255,111,207}
\definecolor{nhilit}{RGB}{128,0,128}  % hilit color in notes
\definecolor{nvhilit}{RGB}{255,0,128} % vhilit for notes
\definecolor{lolit}{RGB}{155,155,155}

\newcommand{\hilit}{\color{hilit}}
\newcommand{\vhilit}{\color{vhilit}}
\newcommand{\nhilit}{\color{nhilit}}
\newcommand{\nvhilit}{\color{nvhilit}}
\newcommand{\lolit}{\color{lolit}}

% use those colors
\setbeamercolor{titlelike}{fg=title}
\setbeamercolor{subtitle}{fg=subtitle}
\setbeamercolor{institute}{fg=gray}
\setbeamercolor{normal text}{fg=foreground,bg=background}
\setbeamercolor{item}{fg=foreground} % color of bullets
\setbeamercolor{subitem}{fg=gray}
\setbeamercolor{itemize/enumerate subbody}{fg=gray}
\setbeamertemplate{itemize subitem}{{\textendash}}
\setbeamerfont{itemize/enumerate subbody}{size=\footnotesize}
\setbeamerfont{itemize/enumerate subitem}{size=\footnotesize}

% page number
\setbeamertemplate{footline}{%
    \raisebox{5pt}{\makebox[\paperwidth]{\hfill\makebox[20pt]{\lolit
          \scriptsize\insertframenumber}}}\hspace*{5pt}}

% add a bit of space at the top of the notes page
\addtobeamertemplate{note page}{\setlength{\parskip}{12pt}}

% default link color
\hypersetup{colorlinks, urlcolor={hilit}}

% a few macros
\newcommand{\bi}{\begin{itemize}}
\newcommand{\bbi}{\vspace{24pt} \begin{itemize} \itemsep8pt}
\newcommand{\ei}{\end{itemize}}
\newcommand{\ig}{\includegraphics}
\newcommand{\subt}[1]{{\footnotesize \color{subtitle} {#1}}}
\newcommand{\ttsm}{\tt \small}
\newcommand{\figh}[2]{\centerline{\includegraphics[height=#2\textheight]{#1}}}
\newcommand{\figw}[2]{\centerline{\includegraphics[width=#2\textwidth]{#1}}}

%%%%%%%%%%%%%%%%%%%%%%%%%%%%%%%%%%%%%%%%%%%%%%%%%%%%%%%%%%%%%%%%%%%%%%
% end of header
%%%%%%%%%%%%%%%%%%%%%%%%%%%%%%%%%%%%%%%%%%%%%%%%%%%%%%%%%%%%%%%%%%%%%%

\title{Version control}
\subtitle{with git and GitHub}
\author{\href{http://www.biostat.wisc.edu/~kbroman}{Karl Broman}}
\institute{Biostatistics \& Medical Informatics, UW{\textendash}Madison}
\date{\href{http://www.biostat.wisc.edu/~kbroman}{\tt \scriptsize \color{white} biostat.wisc.edu/{\textasciitilde}kbroman}
\\[-4pt]
\href{http://github.com/kbroman}{\tt \scriptsize \color{white} github.com/kbroman}
\\[-4pt]
\href{https://twitter.com/kwbroman}{\tt \scriptsize \color{white} @kwbroman}
\\[-4pt]
{\scriptsize Course web: \href{http://bit.ly/tools4rr}{\tt bit.ly/tools4rr}}
}

\begin{document}

{
\setbeamertemplate{footline}{} % no page number here
\frame{
  \titlepage

\vfill
\hfill {\color{lolit} \scriptsize
Slides prepared with \href{http://pages.cs.wisc.edu/~syounkin/}{Sam Younkin}}

\note{Version control is not strictly necessary for reproducible
  research, and it's admittedly a lot of work (to learn and to use) in
  the short term, but the long term benefits are enormous.

  The advantages are: you'll save the entire history of changes to a
  project, you can go back to any point in time (and see what has
  changed between any two points in time), you don't have to worry
  about breaking things that work, and you can easily merge changes
  from multiple people.

  I now use version control for basically everything: software, data
  analysis projects, papers, talks, and web sites.

  People are more resistant to version control than to any other
  tool, because of the short-term effort and the lack of recognition
  of the long-term benefits.
}
} }


\begin{frame}[c]{}

% comic from http://www.phdcomics.com/comics/archive.php?comicid=1531
\centerline{\includegraphics[height=3.2in]{Images/phd101212s.png}}

\vfill
\color{gray} \tiny
\centerline{\url{http://www.phdcomics.com/comics/archive.php?comicid=1531}}

\note{This is typical. And never use ``final'' in a file name.
}
\end{frame}

\begin{frame}{\only<1>{Methods for tracking versions}\only<2|handout 0>{Suppose it stops working\dots}}
\bbi
\item Don't keep track
\onslide<2>{
\bi
\item good luck!
\ei
}
\item Save numbered zip files
\onslide<2>{
\bi
\item Unzip versions and {\tt diff}
\ei
}
\item Formal version control
\onslide<2>{
\bi
\item Easy to study changes back in time
\item Easy to jump back and test
\ei
}
\ei

\note{There are three methods for keeping track of changes: don't keep
  track, periodically zip/tar a directory with a version number, or
  use formal version control.
  
  Imagine that some aspect of your code has stopped working at some
  point. You know it was working in the past, but it's not working
  now. How easy is it to figure out where the problem was introduced?
}
\end{frame}

\begin{frame}{Why use formal version control?}
\bbi
\item History of changes
\item Able to go back
\item No worries about breaking things that work
\item Merging changes from multiple people
\ei

\note{With formal version control, you'll save the entire history of
  changes to the project, and you can easily go back to any
  point in the history of the project, to see how things were behaving
  at that point.

  You'll be able to make modifications (e.g., to try out a new
  feature) without worrying about breaking things that work.

  And version control is especially useful for collaboration. If a
  collaborator has made a bunch of changes, it'll be much easier to
  see what was changed and to incorporate those changes.
}
\end{frame}

\begin{frame}[c]{Example repository}

\figh{Images/example_repo}{0.80}

\onslide<2|handout 0>{
\vspace*{-0.65\textheight}
\figh{Images/example_repo_zoom}{0.55}
}

\note{This is a snapshot of a repository on GitHub: a set of files and
  subdirectories with more files. You can easily explore the contents.
}
\end{frame}


\begin{frame}[c]{Example history}

\figh{Images/example_history}{0.80}

\note{This is a short view of the history of changes to the
  repository: a series of ``commits.''
}
\end{frame}

\begin{frame}[c]{Example commit}

\figh{Images/example_commit}{0.80}
\note{This is an example of one of those commits, highlighting what
  lines were added and what lines were removed.
}
\end{frame}




\begin{frame}{What is git?}
\bbi
\item Formal version control system
\item Developed by Linus Torvalds (developer of Linux)
\bi
\item used to manage the source code for Linux
\ei
\item Tracks any content (but mostly plain text files)
\bi
\item source code
\item data analysis projects
\item manuscripts
\item websites
\item presentations
\ei
\ei

\note{We're going to focus on git, the version control system
  developed by Linus Torvalds for managing the source code for Linux.

  You can track any content, but it's mostly for tracking plain text
  files, but that can be most anything (source code, data analysis
  projects, manuscripts, websites, presentations).
}
\end{frame}


\begin{frame}{Why use git?}
\bbi
\item It's fast
\item You don't need access to a server
\item Amazingly good at merging simultaneous changes
\item Everyone's using it
\ei

\note{Git is fast, you can use it locally on your own computer, it's
  amazingly good at merging changes, and there are lots of people
  using it.
}
\end{frame}

\begin{frame}{What is GitHub?}
\bbi
\item A home for git repositories
\item Interface for exploring git repositories
\item {\color{hilit} Real} open source
\bi
\item immediate, easy access to the code
\ei
\item Like facebook for programmers
\item Free 2-year "micro" account for students
\bi
\item \href{http://education.github.com}{github.com/edu}
\ei
\item (Bitbucket.org is an alternative)
\bi
\item free private repositories
\ei
\ei

\note{GitHub is a website that hosts git repositories, with a nice
  graphical user interface for exploring git repositories.

  Source code on GitHub is \emph{really\/} open source: anyone can
  study it and grab it.

  GitHub is sort of like Facebook for programmers: you can see what
  people are up to, and easily collaborate on shared projects.

  It's free to have public repositories on GitHub; if you want private
  repositories, you generally have to pay, but I understand that
  students can get a two-year account that allows 5 private
  repositories.

  Bitbucket.org is an alternative; it allows unlimited private
  repositories. I'm cheap, so I use Bitbucket for my private
  repositories.
}
\end{frame}

\begin{frame}{Why use GitHub?}
\bbi
\item It takes care of the server aspects of git
\item Graphical user interface for git
\bi
\item Exploring code and its history
\item Tracking issues
\ei
\item Facilitates:
\bi
\item Learning from others
\item Seeing what people are up to
\item Contributing to others' code
\ei
\item Lowers the barrier to collaboration
\bi
\item "There's a typo in your documentation." vs. \\
"Here's a correction for your documentation."
\ei
\ei

\note{GitHub takes care of the server aspects of git, and you get a
  great GUI for exploring your repositories.

  GitHub is great for browsing others' code, for learning; you don't
  even have to download it to your computer. And it's really easy to
  contribute to others' code (e.g., to report typos in their
  documentation).
}
\end{frame}



\begin{frame}
\frametitle{Basic use}

\vspace{-18pt}

\bbi
\item Change some files
\item See what you've changed
\bi
\item[] {\tt git status}
\item[] {\tt git diff}
\item[] {\tt git log}
\ei
\item Indicate what changes to save
\bi
\item[] {\tt git add}
\ei
\item Commit to those changes
\bi
\item[] {\tt git commit}
\ei
\onslide<2->{
\item Push the changes to GitHub
\bi
\item[] {\tt git push}
\ei }
\onslide<3->{
\item Pull changes from your collaborator
\bi
\only<3>{
\item[] {\tt git pull}
}
\onslide<4>{
\item[] {\tt git fetch}
\item[] {\tt git merge}
}
\ei }
\ei

\note{These are the basic git commands you'll use day-to-day.

  {\tt git status} to see the current state of things, 
  {\tt git diff} to see what's changed, and {\tt git log} to look at
  the history.

  After you've made some changes, you'll use {\tt git add} to indicate
  which changes you want to commit to, and {\tt git commit} to commit
  to them (to add them to the repository).

  You use {\tt git push} to push changes to GitHub, and {\tt git pull}
  (or {\tt git fetch} and {\tt git merge}) to pull changes from a
  collaborator's repository, or if you're synchronizing a repository
  between two computers.
}
\end{frame}

\begin{frame}[fragile]
\frametitle{Initialize repository}
\bbi
\item Create {\lolit (and {\tt cd} to)} a working directory
\bi
\item For example, \verb|~/Docs/Talks/Graphs|
\ei
\item Initialize it to be a git repository
\bi
\item {\tt \color{hilit} git init}
\item Creates subdirectory \verb|~/Docs/Talks/Graphs/.git|
\ei
\ei

\begin{lstlisting}
$ mkdir ~/Docs/Talks/Graphs
$ cd ~/Docs/Talks/Graphs
$ git init
Initialized empty Git repository in ~/Docs/Talks/Graphs/.git/
\end{lstlisting}

\note{If you're starting a new, fresh project, you make a directory
  for it and go into that directory, and then you type {\tt git
    init}. This creates a {\tt .git} subdirectory.
}
\end{frame}

\begin{frame}[fragile]
\frametitle{Produce content}
\bbi
\item Create a {\tt README.md} file
\ei

\bigskip 
\begin{lstlisting}
## Talk on &ldquo;How to display data badly&rdquo;

These are slides for a talk that I give as often as possible,
because it's fun.

This was inspired by Howard Wainer's article, whose title I
stole: H Wainer (1984) How to display data badly.
American Statistician 38:137-147

A recent PDF is
[here](
http://www.biostat.wisc.edu/~kbroman/talks/graphs2013.pdf).
\end{lstlisting}

\note{Start creating a bit of content, such as a Readme file. You can
  use Markdown to make it look nicer.
}
\end{frame}


\begin{frame}[fragile]
\frametitle{Incorporate into repository}
\bbi
\item Stage the changes using {\tt \color{hilit} git add}
\ei

\begin{lstlisting}
$ git add README.md
\end{lstlisting}

\note{Use {\tt git add} to tell git that you want to start keeping
  track of this file.  This is called ``staging,'' or you say the file
  is ``staged.''
}
\end{frame}

\begin{frame}[fragile]
\frametitle{Incorporate into repository}
\bbi
\item Now commit using {\tt \color{hilit} git commit}
\ei

\begin{lstlisting}
$ git commit -m "Initial commit of README.md file"
[master (root-commit) 32c9d01] Initial commit of README.md file
 1 file changed, 14 insertions(+)
 create mode 100644 README.md
\end{lstlisting}

\bi
\item The \texttt{-m} argument allows one to enter a message
\item Without \texttt{-m}, \texttt{git} will spawn a text editor
\item Use a meaningful message
\item Message can have multiple lines, but make 1st line an overview
\ei

\note{Use {\tt git commit} to add the file to the repository.
}
\end{frame}




\begin{frame}{Using git on an existing project}

\bbi
\item {\tt git init}
\item Set up {\tt .gitignore} file
\item {\tt git status} {\footnotesize \color{lolit} (did you miss any?)}
\item {\tt git add .} {\footnotesize \color{lolit} (or name files individually)}
\item {\tt git status} {\footnotesize \color{lolit} (did you miss any?)}
\item {\tt git commit}
\ei


\note{I recommend using git with all of your current projects.
  Start with one.

  Go into the directory and type {\tt git init}. Then use {\tt git
    add} repeatedly, to indicate which files you want to add to the
  repository.

  Then use {\tt git commit} to make an initial commit.
}
\end{frame}



\begin{frame}[fragile]{Removing/moving files}

\vspace{24pt}

For files that are being tracked by git:

\bigskip

\hspace{1em} Use {\tt \color{hilit} git rm} instead of just {\tt rm}

\hspace{1em} Use {\tt \color{hilit} git mv} instead of just {\tt mv}

\bigskip

\begin{lstlisting}
$ git rm myfile
$ git mv myfile newname
$ git mv myfile SubDir/
$ git commit
\end{lstlisting}

\note{For files that are being tracked by git: If you want to change
  the name of a file, or if you want to move it to a subdirectory, you
  can't just use {\tt mv}, you need to use {\tt git mv}.

  If you want to remove a file from the project, don't use just {\tt
    rm}, use {\tt git rm}. Note that the file won't be
  \emph{completely\/} removed; it'll still be within the history.
}
\end{frame}




\begin{frame}[fragile]{A few points on commits}
\bbi
\item Use frequent, small commits
\item Don't get out of sync with your collaborators
\item Commit the sources, not the derived files
\bi
\item[] (R code not images)
\ei
\item Use a {\tt .gitignore} file to indicate files to be ignored
\ei

\begin{lstlisting}
*~
manuscript.pdf
Figs/*.pdf
.RData
.RHistory
*.Rout
*.aux
*.log
*.out
\end{lstlisting}

\note{
}
\end{frame}



\begin{frame}[fragile]{Getting started with GitHub}
\bbi
\item Get an account
\item Set up ssh keys
\bi
\item Look for files {\color{hilit} \verb|~/.ssh/id_rsa|} and {\color{hilit} \verb|~/.ssh/id_rsa.pub|}
\item {\color{hilit} \verb|ssh-keygen |}\verb|-t rsa -C "your_email@example.com"|
\item Copy contents of \verb|~/.ssh/id_rsa.pub|
\ei
\item Add SSH key at GitHub
\bi
\item Account settings
\item SSH Keys
\item Add SSH key
\item Paste contents of \verb|~/.ssh/id_rsa.pub|
\ei
\item Similar thing at BitBucket
\ei


\note{
}
\end{frame}

\begin{frame}{Set up GitHub repository}

\only<1>{
\bbi
\item Click the "Create a new repo" button
\item Give it a {\color{hilit} name} and description
\item Click the "Create repository" button
\item Back at the command line:
  \bi
  \item[] {\tt git remote add origin git@github.com:username/{\color{hilit} repo}}
  \item[] {\tt git push -u origin master}
  \ei
\ei
}
\only<2->{\vspace{24pt}}
\only<2 | handout 0>{\figw{Images/new_repo_1.png}{0.95}}
\only<3 | handout 0>{\figw{Images/new_repo_2.png}{0.95}}


\note{
}
\end{frame}

\begin{frame}[fragile]{Configuration file}

\vspace{24pt}

Part of a {\tt .git/config} file:

\begin{lstlisting}
[remote "origin"]
	url = git@github.com:kbroman/qtl.git
	fetch = +refs/heads/*:refs/remotes/origin/*

[branch "master"]
	remote = origin
	merge = refs/heads/master

[remote "brian"]
	url = git://github.com/byandell/qtl.git
	fetch = +refs/heads/*:refs/remotes/brian/*
\end{lstlisting}

\note{
}
\end{frame}


\begin{frame}{Issues and pull requests}
\bbi
\item Problem with or suggestion for someone's code?
\bi
\item Point it out as an Issue
\ei
\item Even better: Provide a fix
\bi
\item Fork
\item Clone
\item Modify
\item Commit
\item Push
\item Submit a Pull Request
\ei
\ei

\note{
}
\end{frame}


\begin{frame}[fragile]{Suggest a change to a repo}
\bbi
\item Go to the repository:
\bi
\item[] \verb|http://github.com/someone/repo|
\ei
\item {\color{hilit} Fork} the repository
\bi
\item[] Click the "Fork" button
\ei
\item {\color{hilit} Clone} your version of it
\bi
\item[] {\tt git clone git@github.com:username/repo}
\ei
\item Change things locally, {\tt git \color{hilit} add}, {\tt git \color{hilit} commit}
\item Push your changes to \emph{your\/} GitHub repository
\bi
\item[] {\tt git \color{hilit} push}
\ei
\item Go to \emph{your\/} GitHub repository
\item Click "{\color{hilit} Pull Requests}" and "New pull request"
\ei

\note{
}
\end{frame}


\begin{frame}[fragile]{Pulling a friend's changes}
\bbi
\item Add a connection
\bi
\item[] {\tt git remote add friend git://github.com/friend/repo}
\ei
\item Pull the changes
\bi
\item[] {\tt git pull friend master}
\ei
\item Push them back to your GitHub repo
\bi
\item[] {\tt git push}
\ei
\ei

\note{
}
\end{frame}


\begin{frame}[fragile]{Merge conflicts}

\vspace{12pt}

Sometimes after {\color{hilit} \tt git pull friend master}

\begin{lstlisting}
Auto-merging README.md
CONFLICT (content): Merge conflict in README.md
Automatic merge failed; fix conflicts and then commit the result.
\end{lstlisting}

Inside the file you'll see:

\begin{lstlisting}
<<<<<<< HEAD
A line in my file.
=======
A line in my friend's file
>>>>>>> 031389f2cd2acde08e32f0beb084b2f7c3257fff
\end{lstlisting}

Edit, add, commit, push, submit pull request.

\note{
}
\end{frame}


\begin{frame}{git/GitHub with RStudio}

\vspace{24pt}

\figw{Images/RStudio04.png}{0.90}

\vspace{64pt}

\hfill 
{\small \lolit 
See \href{http://www.biostat.wisc.edu/~kbroman/presentations/GitPrimer.pdf}{GitPrimer.pdf}
or
\href{http://www.rstudio.com/ide/docs/version_control/overview}{RStudio page}}

\note{
}
\end{frame}

\begin{frame}[c]{Delete GitHub repo}

\figh{Images/RStudio12.png}{0.80}

\note{
}
\end{frame}


\begin{frame}{}

\vspace{25mm}

Open source means everyone can see my stupid mistakes.

\vspace{5mm}

Version control means everyone can see every stupid mistake I've ever
made.

\vspace{33mm}
\centerline{\scriptsize \tt \color{gray} \href{http://bit.ly/stupidcode}{bit.ly/stupidcode}}

\note{
}
\end{frame}

\end{document}
